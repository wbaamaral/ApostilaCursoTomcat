\appendix
\section{Configurar instância com serviço}\label{apendice:servico}

Em primeiro lugar, ao tornar o Tomcat um serviço, ele será iniciado automaticamente com o sistema operacional, garantindo que sua aplicação Java esteja sempre disponível, mesmo após um reinício do servidor. Isso significa que você não precisará iniciar manualmente o Tomcat toda vez que precisar executar sua aplicação, economizando tempo e esforço.

Além disso, ao utilizar o Tomcat como um serviço, você pode facilmente controlá-lo através de comandos padrão do sistema, como "start", "stop" e "restart". Isso torna o gerenciamento do Tomcat mais fácil e acessível, mesmo para usuários menos experientes.

Outra vantagem importante é a segurança. Ao executar o Tomcat como um serviço, ele é executado com permissões limitadas, o que ajuda a proteger o sistema operacional contra possíveis ataques ou vulnerabilidades de segurança em sua aplicação Java.

Por fim, utilizar o Tomcat como um serviço também facilita a integração com outras ferramentas de gerenciamento e monitoramento, permitindo que você monitore o desempenho da sua aplicação Java e tome medidas proativas para evitar problemas antes que eles ocorram.

Após criar os dois arquivos, é necessário efetuar um reload na lista de serviços do sistema, e por eles para inicio automático, só para lembra esses comandos são de nível administrativo use o root ou sudo para executa-los.

\begin{verbatim}
# reload serviços
systemctl daemon-reload

# iniciar serviços
systemctl start tomcat1
systemctl start tomcat2

# parar serviços
systemctl stop tomcat1
systemctl stop tomcat2

# reiniciar serviços
systemctl restart tomcat1
systemctl restart tomcat2

# habilitar início automático
systemctl enable tomcat1
systemctl enable tomcat2

# desabilitar início automático
systemctl disable tomcat1
systemctl disable tomcat2

\end{verbatim}

\begin{figure}[H]
	\centering
	\caption[tomcat1.service]{/etc/systemd/system/tomcat1.service}
	\includegraphics[width=0.85\linewidth]{img/code-img-tomcat1}
	\label{fig:tomcat1.service}
\end{figure}

\begin{figure}[H]
	\centering
	\caption[tomcat2.service]{/etc/systemd/system/tomcat2.service}
	\includegraphics[width=0.85\linewidth]{img/code-img-tomcat2}
	\label{fig:tomcat2.service}
\end{figure}
	
\begin{lstlisting}[breaklines=true,basicstyle=\scriptsize,	breaklines=true, label=servico01,showspaces=false,showstringspaces=false,caption={Script exemplo para gerar arquivo de serviço}]
cat >/etc/systemd/system/tomcat1.service <<TERMINOU
[Unit]
Description=Apache Tomcat Web Application Container
After=network.target


[Service]
Type=forking


Environment=JAVA_HOME=$(update-java-alternatives -l|awk '{print $3}')
Environment=CATALINA_PID=/opt/tomcat/temp/tomcat.pid
Environment=CATALINA_HOME=/opt/tomcat
Environment=CATALINA_BASE=/opt/tomcat
Environment='CATALINA_OPTS=−Xms512M −Xmx1024M −server −XX:+UseParallelGC'
Environment='JAVA_OPTS=−Djava.awt.headless=true −Djava.security.egd=file:/dev/./urandom'

ExecStart=/opt/tomcat/bin/startup.sh
ExecStop=/opt/tomcat/bin/shutdown.sh



User=tomcat
Group=tomcat
UMask=0007
RestartSec=10
Restart=always


[Install]
WantedBy=multi−user.target



#
TERMINOU
\end{lstlisting}

\section{Personalizar identificação das vms}

Seu texto de anexo aqui...
