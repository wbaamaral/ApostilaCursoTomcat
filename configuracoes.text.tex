%%%%%%%%%%%%%%%%%%%%%%%%%%%%%%%%%%%%%%%%%%%%%%%%%%%%%%%%%%%%%%%%%%%%%%%%%%%%%%%%%%%%
%                                                                                  %
%                           Configuração de Pacotes                                %
%                                                                                  %
%%%%%%%%%%%%%%%%%%%%%%%%%%%%%%%%%%%%%%%%%%%%%%%%%%%%%%%%%%%%%%%%%%%%%%%%%%%%%%%%%%%%
\usepackage[brazilian]{babel} % pacote para a internacionalização de documentos, neste caso configurado para o português do Brasil;
\usepackage[utf8]{inputenc} % permite a entrada de caracteres especiais (como acentos) diretamente no código fonte;
\usepackage[T1]{fontenc} % define a codificação de caracteres a ser usada na saída;
\usepackage{fontspec} % permite o uso de fontes instaladas no sistema;


%%%%%%%%%%%%%%%%%%%%%%%%%%%%%%%%%%%%%%%%%%%%%%%%%%%%%%%%%%%%%%%%%%%%%%%%%%%%%%%%%%%%
%                                                                                  %
%                           Configuração de Fontes                                 %
%                                                                                  %
%%%%%%%%%%%%%%%%%%%%%%%%%%%%%%%%%%%%%%%%%%%%%%%%%%%%%%%%%%%%%%%%%%%%%%%%%%%%%%%%%%%%

%\setmainfont{Times New Roman} % é usado para definir a fonte principal do documento quando se está usando o pacote fontspec. 

% Defina uma fonte secundária para títulos
\newfontfamily\ubuntu{Ubuntu}
\newfontfamily\tty{Herr Von Muellerhoff}
\newfontfamily\firacoderetina{Fira Code Retina}

\usepackage{titling} % O pacote "titling" fornece comandos para personalização de títulos em documentos, como \maketitle, \title, \author, \date, entre outros. Ele permite que os usuários ajustem a posição, o espaçamento e o estilo de cada elemento do título.

\usepackage{fancyhdr} % permite a customização dos cabeçalhos e rodapés das páginas;

\usepackage{nameref} % permite referenciar nomes de seções, figuras, tabelas, etc. pelo seu nome ao invés do número;

\usepackage{tocloft} % fornece recursos para personalizar o formato de listas de tabelas, figuras e outras listas no LaTeX. 

\usepackage{geometry} %pacote para configuração das margens e tamanho do papel do documento;
\geometry{a4paper,left=2cm,right=1cm,top=2cm,bottom=2cm} % Configuração de margens

\usepackage{xcolor} % permite o uso de cores no documento.
\usepackage{graphicx} % pacote para a inclusão de imagens em formatos diversos;
\usepackage{wrapfig} % utilizado para criar ambiente de flutuação do logotipo

\usepackage{indentfirst} %O pacote indentfirst garante que o primeiro parágrafo de cada seção também tenha recuo, e não apenas os parágrafos seguintes. O comprimento \parindent define o tamanho do  recuo e pode ser ajustado para o valor desejado.
\setlength{\parindent}{1.5cm}

\usepackage{listings} % pacote para exibição de código fonte com formatação personalizada;
\renewcommand{\lstlistingname}{Arquivo}
\renewcommand{\lstlistlistingname}{Lista de Arquivos e Configurações}
\newlistof{lstlistoflistings}{lol}{\lstlistlistingname}
\lstset{
	language=sh,
	basicstyle=\small\ttfamily,
	numbers=left,
	numberstyle=\tiny\color{black},
	frame=single,
	rulecolor=\color{gray!20},
	backgroundcolor=\color{gray!5},
	keywordstyle=\color{blue},
	stringstyle=\color{purple},
	commentstyle=\color{olive},
	morekeywords={mkdir,cd,ls,echo,apt},
	}

\usepackage{identkey} % O pacote identkey define um ambiente para criar uma lista de itens com identificadores alfanuméricos personalizados. É útil para criar referências cruzadas e outros tipos de documentos que precisam de identificadores personalizados para itens.

\usepackage{lipsum} % para gerar texto aleatório
\usepackage{enumitem} % para configurar o recuo

\usepackage[colorlinks=true, linkcolor=darkgray]{hyperref}
\hypersetup{
	colorlinks=true,
	linkcolor=darkgray,
	filecolor=magenta,      
	urlcolor=black,
	pdftitle={Treinamento - Administração Apache Tomcat},
	pdfpagemode=FullScreen,
}

\usepackage{float} % O pacote float fornece melhor controle sobre os ambientes flutuantes (como figuras, tabelas e outros objetos) no LaTeX, permitindo que você personalize a posição deles na página. Ele oferece opções adicionais para especificar onde o objeto flutuante deve ser colocado, como "here" (aqui), "top" (topo), "bottom" (fundo) e "page" (página). Além disso, ele permite definir a posição padrão para um determinado tipo de objeto flutuante em todo o documento.

\cftsetindents{section}{1.5cm}{0.5cm} % Define o recuo do título de seção e da numeração em relação às margens esquerda e direita do índice.
\cftsetpnumwidth{1cm} % Define a largura da coluna da numeração no índice.
\usepackage{tabularx} % Permite a criação de tabelas com colunas de largura automática.
\newcolumntype{C}{>{\centering\arraybackslash}p{4cm}}  % Define um novo tipo de coluna "C" com largura fixa de 4cm e alinhamento centralizado.
\newcolumntype{D}{>{\raggedleft\arraybackslash}p{2cm}} % Define um novo tipo de coluna "D" com largura fixa de 2cm e alinhamento à direita.

\usepackage{accsupp}
% define a função de cópia
\newcommand{\CopyAndPaste}{%
	\def\lst@OutputSpace{\lst@ttfamily\char32\protect\BeginAccSupp{method=hex,unicode,ActualText=20}\lst@ActualOutputSpace\protect\EndAccSupp{}\lst@ttfamily\char32}%
	\let\lst@OutputTab=\lst@OutputSpace
	\let\lst@OutputCR=\lst@OutputSpace
	\let\lst@OutputFF=\lst@OutputSpace
}


\title{\ubuntu Administração Apache Tomcat 9.0} % Define o título do documento.
\author{\emph{Autor}: Murilo Bavaro Bilia} % Define o autor do documento.
\date{Fevereiro de 2023} %Define a data do documento. Se deixado em branco, a data atual será utilizada.

%alinhamento de sumário.
\usepackage{titletoc}

\titlecontents{chapter}
[0pt]
{\addvspace{1em}}
{\bfseries\chaptername\ \thecontentslabel\quad}
{}
{\hfill\contentspage}

\titlecontents{section}
[1.5em]
{\addvspace{0.5em}}
{\contentslabel{1.5em}\hspace{0.5em}}
{\hspace{0.5em}}
{\titlerule*[0.5pc]{.}\contentspage}

\titlecontents{subsection}
[3em]
{\addvspace{0.25em}}
{\contentslabel{2.25em}\hspace{0.75em}}
{\hspace{0.75em}}
{\titlerule*[0.5pc]{.}\contentspage}
