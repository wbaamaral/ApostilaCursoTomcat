\section{Instalação do Java Development Kit}

Instalação automática via sistema operacional:


\begin{lstlisting}[breaklines=true,basicstyle=\firacoderetina,
breaklines=true,caption=\firacoderetina Instalando pacotes openjdk,
postbreak=\mbox{\textcolor{red}{$\hookrightarrow$}\space},
showstringspaces=false]

vm01:/home/#		
vm01:/home/# 		
vm01:/home/# apt update && apt install openjdk-17-jdk -y 
vm01:/home/#

\end{lstlisting}	

Instalação do Oracle JDK via alternatives:

\begin{list}{•}{}
	\setlength{\leftskip}{1.5cm}
	\setlength{\rightskip}{0pt plus 1.0cm}
	\setlength{\parindent}{-1.5cm}
\item Fazer o Download do JDK no site da Oracle
\item Extrair o conteúdo do arquivo baixado no diretório /usr/lib/jvm
\item Executar o comando abaixo para fazer a configuração via alternatives:
\item update-alternatives --install /usr/bin/java java /usr/lib/jvm/jdk<versão>/bin/java 1
\item update-alternatives --set java /usr/lib/jvm/jdk<versão>/bin/java
\item update-alternatives --config java
\end{list}

Instalação do Oracle JDK manualmente:

\begin{list}{•}{}
	\setlength{\leftskip}{1.5cm} 
	\setlength{\rightskip}{0pt plus 1.0cm}
	\setlength{\parindent}{-1.5cm}
\item Fazer o Download do JDK no site da Oracle
\item Extrair o conteúdo do arquivo baixado no diretório /usr/lib/jvm
\item Criar um link simbólico do binário do Java em /etc/alternatives:\newline
	{\small \texttt{ln -s /usr/lib/jvm/jdk<versão>/bin/java /etc/alternatives/java}}
\item Criar um link simbólico do alternatives para o diretório /usr/bin:\newline
	{\small \texttt{ln -s /etc/alternatives/java /usr/bin/java}}
\end{list}
