\section{Instalação do Sistema Operacional}

Todo o ambiente de treinamento será montado dentro do hypervisor VirtualBox.
O laboratório será composto por 4 Virtual Machines executando o sistema operacional Debian\footnote{\href{https://www.debian.org/CD/http-ftp/}{Baixando as imagens do CD/DVD Debian via HTTP/FTP}} 11.

Cada Virtual Machine terá a seguinte função:
\begin{table}[h]
	\caption{Definição do laboratório: Nomes, Funções e Endereços}
	\centering
	\begin{tabularx}{0.69\textwidth}{D|X|C}
		\firacoderetina\textbf{VM1} 	& \ubuntu Tomcat1      				& \ubuntu 192.168.5.191 \\ \hline
		\firacoderetina\textbf{VM2}    & \ubuntu Tomcat2      				& \ubuntu 192.168.5.192 \\ \hline
		\firacoderetina\textbf{FRONT}  & \ubuntu Servidor Web 				& \ubuntu 192.168.5.193 \\ \hline
		\firacoderetina\textbf{DB}  	& \ubuntu Servidor Banco de dados 	& \ubuntu 192.168.5.194 \\ \hline
	\end{tabularx}
\end{table}

As configurações minímas de cada VM serão as seguintes:

\begin{list}{•}{}
	\setlength{\leftskip}{1.8cm} 
	\setlength{\rightskip}{0pt plus 1.0cm}
	\setlength{\parindent}{-1.8cm}
\item \textbf{VM1} - 1GB RAM, 1 CPU, 8 GB HD
\item \textbf{VM2} - 1GB RAM, 1 CPU, 8 GB HD
\item \textbf{FRONT} - 512MB RAM, 1 CPU, 8 GB HD
\item \textbf{DB} - 512MB RAM, 1 CPU, 8 GB HD
\end{list}

O particionamento do disco ficará da seguinte forma:
\begin{table}[H]
	\caption{Sugestão de particionamento}
	\centering
	\begin{tabularx}{0.4\textwidth}{||C|X||}
		\hline
		\textbf{/}     & 3072 MB \\ \hline
		\textbf{/boot} & 1024 MB \\ \hline
		\textbf{/var}  & 1024 MB \\ \hline
		\textbf{/srv}  & 2048 MB \\ \hline
		\textbf{/tmp}  & 1024 MB \\ \hline
		\textbf{Swap}  &  512 MB \\ \hline
	\end{tabularx}
\end{table}	

Após a instalação do Sistema Operacional, será necessário instalar os pacotes de software abaixo que irão nos auxiliar durante o treinamento:

\begin{lstlisting}[breaklines=true,basicstyle=\firacoderetina,breaklines=true,
caption=\firacoderetina Instalando pacotes extras, postbreak=\mbox{\textcolor{red}{$\hookrightarrow$}\space},
showstringspaces=false]

# Instalando pacotes extras
apt update && apt install -y net-tools  vim sysstat iotop bind9utils zip unzip tcpdump figlet htop linuxlogo

\end{lstlisting}


Em seguida, vamos adicionar um usuário para executar o Tomcat:

\begin{lstlisting}[breaklines=true,basicstyle=\firacoderetina,
breaklines=true,caption=\firacoderetina Criando usuário e grupo,
postbreak=\mbox{\textcolor{red}{$\hookrightarrow$}\space},
showstringspaces=false]

vm01:/home/# 		
vm01:/home/# groupadd tomcat		
vm01:/home/# useradd -s /bin/false -g tomcat -d /srv/tomcat tomcat
vm01:/home/# 		

\end{lstlisting}
